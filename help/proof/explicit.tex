\doc{Proofs by explicit commands}
\ref{explicit-commands}

The special proof method \fq{explicit-commands} directs LP not to apply any
method of backward inference automatically to a conjecture, but to wait for an
explicit method to be given with a subsequent \cflink{resume} command.  This
method is useful in several situations.
\begin{itemize}
\item
It can be used to prevent LP from attempting to normalize a conjecture when it
would be time-consuming and unfruitful to do so.  For example, if a conjecture
includes many conjuncts, it may be appropriate to first compute some critical
pairs, then apply the \fq{/\}-method, and finally normalize the individual
subgoals.
\p
\item
It can be used to delay the start of a proof until the user has had an
opportunity to \cflink{make} the conjecture immune, so that it is not reduced
once it has been proved.
\end{itemize}
