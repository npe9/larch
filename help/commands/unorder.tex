\doc{The unorder command}
\ref{unorder-command}

The \def{unorder command} turns rewrite rules back into formulas.

\head{2}{\dlink{../symbols/syntax}{Syntax}}
\begin{verbatim}
\sd{unorder-command} ::= \f{unorder} [ \slink{../misc/names}{names} ]
\end{verbatim}

\head{2}{Examples}
\begin{verbatim}
unorder
unorder nat
\end{verbatim}

\head{2}{Usage} 
The \fq{unorder} command causes LP to turn the named rewrite rules back into
formulas.  If no names are specified, LP unorders all rewrite rules.
\p
Even if the \setlink{automatic-ordering} setting is \fq{on}, the unordered
formulas are not immediately reordered into rewrite rules.  This gives users an
opportunity to change the \setlink{ordering-method} or the
\dlink{register}{registry}.  The formulas will be oriented into
rewrite rules in response to an explicit \cflink{order} command or in response
to some other command that would invoke LP's automatic ordering (for example,
the \cflink{assert} command).
