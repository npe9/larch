\doc{The execute command}
\ref{execute-command}
\ref{execute-silently}
\ref{file}
\ref{execution}
\ref{executes}
\ref{executing}

The \def{execute command} causes LP to execute commands from a specified file.

\head{2}{\dlink{../symbols/syntax}{Syntax}}
\begin{verbatim}
\sd{execute-command} ::= ( \f{execute} | \f{execute-silently} ) \s{file}
\sd{file}            ::= \slink{../symbols/symbols}{blank-free-string}
\end{verbatim}

\head{2}{Examples}
\begin{verbatim}
execute myProof
\end{verbatim}

\head{2}{Usage} 
The \fq{execute} command causes LP to execute the commands in the file named
\s{file}\f{.lp} (unless \s{file} contains a period, in which case LP does not
supply the suffix \f{.lp}) on the current LP 
\dlink{../settings/lp-path}{search path}.  The \fq{execute} command is 
ordinarily used to execute commands from a file that was created by the
\dflink{../commands/set}{set script} or \cflink{write} commands, but any text
file may be specified.
\p
Further \fq{execute} commands may occur in \fq{.lp} files, but recursive 
\fq{.lp} files are not allowed.  Once a \fq{.lp} file has been exhausted, LP 
resumes accepting input from the previous \fq{.lp} file or from the user if no
other file is being executed.  If LP encounters an error while executing a
file, or if the user \olink{interrupt}{interrupts} LP, LP takes subsequent
input from the terminal.
\p
The \fq{execute-silently} command is just like \fq{execute}, except that it
produces no output.

