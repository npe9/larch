\doc{The make command}
\ref{make-command}
\ref{fact-status}

The \def{make command} changes the \setlink{activity} or \setlink{immunity} of
a collection of facts or of the current conjecture.

\head{2}{\dlink{../symbols/syntax}{Syntax}}
\begin{verbatim}
\sd{make_command} ::= \f{make} \s{fact-status} ( \slink{../misc/names}{names} | \f{conjecture} )
\sd{fact-status}  ::= \f{active} | \f{passive} | \f{immune} | \f{nonimmune} | \f{ancestor-immune}
\end{verbatim}

Note: The \s{fact-status} can be \dlink{../misc/abbreviation}{abbreviated}.

\head{2}{Examples}

\begin{verbatim}
make inactive rewrite-rules
make immune conjecture
\end{verbatim}

\head{2}{Usage}

LP automatically uses any rewrite rules made active by the \fq{make} command to
normalize terms appearing in the current conjecture or in nonimmune facts in
LP's logical system.  It also automatically applies any deduction rules made
active by the \fq{make} command.
\p
LP automatically normalizes any rewrite rules and deduction rules made
nonimmune by the \fq{make} command, and it applies all active deduction rules
to these deimmunized rules.
\p
See the \setlink{activity} and \setlink{immunity} settings.

