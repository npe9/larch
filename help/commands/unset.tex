\doc{The unset command}
\ref{unset-command}

The \def{unset command} returns settings to their default values.

\head{2}{\dlink{../symbols/syntax}{Syntax}}
\begin{verbatim}
\sd{unset-command} ::= \f{unset} ( \slink{set}{setting-name}  | \f{all} )
\end{verbatim}

Note: The argument to the \fq{unset} command can be
\dlink{../misc/abbreviation}{abbreviated}.

\head{2}{examples}
\begin{verbatim}
unset script
unset all
\end{verbatim}
The \fq{unset} command sets the value of \s{setting-name} to its default value.
In particular, \fq{unset script} stops recording user input in a
\setlink{script} file and \fq{unset log} stops recording the session in a
\setlink{log} file.  The \fq{unset all} command sets the value of all settings 
to their defaults.

\head{2}{See also}
\begin{itemize}
\item The \cflink{set} command for a description of the legal settings and their
default values
\end{itemize}

