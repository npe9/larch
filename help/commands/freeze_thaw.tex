\doc{The freeze and thaw commands}
\ref{freeze-command}
\ref{thaw-command}

The \def{freeze} and \def{thaw commands} enable LP to save and restore its
state. 

\head{2}{\dlink{../symbols/syntax}{Syntax}}
\begin{verbatim}
\sd{freeze-command} ::= (\f{freeze} | \f{freeze-current}) \slink{execute}{file}
\sd{thaw-command}   ::= \f{thaw} \slink{execute}{file}
\end{verbatim}

\head{2}{Examples}
\begin{verbatim}
freeze case1
thaw case1
\end{verbatim}

\head{2}{Usage} 
The \fq{freeze} command saves LP's entire state, excluding the
\clink{statistics} and \dlink{set}{global settings}, but
including all proof contexts, in the file named \s{file}\f{.lpfrz} (unless
\s{file} contains a period, in which case LP does not supply the suffix
\f{.lpfrz}) in LP's current working \setlink{directory}.  If a file with that 
name already exists, its previous contents are erased; if it does not exist, it
is created.
\p
The \fq{freeze-current} command does the same, except that it only saves the
state of the current proof context.  This command is faster and uses less disk
space than the \fq{freeze} command.  It is useful primarily for trying
different strategies for proving the current conjecture.
\p
The \fq{thaw} command restores the state of LP from that saved previously using
the \fq{freeze} command in the file named \s{file}\f{.lpfrz} (unless \s{file}
contains a period, in which case LP does not supply the suffix \f{.lpfrz}) on
the current LP \dlink{../settings/lp-path}{search path}.  The \fq{thaw} command
will not thaw files that were frozen using an out-of-date version of LP.
\p
These commands are useful for checkpointing attempted proofs.  They are also
useful for saving and restoring completed or partially-completed systems.  

\head{2}{See also}
\begin{itemize}
\item
\dlink{forget}{Forgetting} information to save space before a \fq{freeze}
\item
\dlink{write}{Writing} commands into a file to recreate the current proof 
context
\end{itemize}


