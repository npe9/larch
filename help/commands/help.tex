\doc{The help command}
\ref{help-command}
\ref{topic}

The \def{help command} provides information about the use of LP.  This same
information can also be viewed using a web browser starting at the home page '
in \fq{~lp/help/introduction.html}.

\head{2}{\dlink{../symbols/syntax}{Syntax}}
\begin{verbatim}
\sd{help-command} ::= \f{help} \s{topic}
\sd{topic}        ::= \slink{../symbols/symbols}{blank-free-string}
\end{verbatim}

\head{2}{Examples}
\begin{verbatim}
help ?
help commands
\end{verbatim}

\head{2}{Usage} 
The \fq{help} command provides a detailed explanation of the requested topic,
which can be specified by an unambiguous prefix.  The command \fq{help ?}
produces a terse list of topics for which help is available.  The command
\fq{help commands} produces a summary of the LP commands.  The command
\fq{help lp} produces a list of general topics for which help is available.
\p
If you don't get information on the topic you expected after typing a command
like \fq{help rewrite}, try typing \fq{help rewrite-} to see if there is
further information about related topics (e.g., \fq{rewrite-command} or
\fq{rewrite-rule}).
