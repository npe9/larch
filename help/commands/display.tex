\doc{The display command}
\ref{display-command}
\ref{information-type}

The \def{display command} displays information about LP's logical system.

\head{2}{\dlink{../symbols/syntax}{Syntax}}
\begin{verbatim}
\sd{display-command}  ::= \f{display} [ \s{information-type} ] [ \slink{../misc/names}{names} ]
\sd{information-type} ::= \f{classes} | \f{conjectures} | \f{facts} | \f{names}
                          | \f{ordering-constraints} | \f{proof-status} | \f{symbols}
\end{verbatim}

\head{2}{Examples}
\begin{verbatim}
display
display *Hyp
display ordering-constraints contains-operator(+)
\end{verbatim}

\head{2}{Usage} 
The \fq{display} command displays information of the requested
\s{information-type} about all facts and conjectures with names matching
\s{names}.  If \s{information-type} is omitted, it is assumed to be \fq{facts}.
\begin{description}
\dt \f{display classes}
\dd
Displays the definitions of all \slink{../misc/class}{class-name}s.
\p
\dt \f{display conjectures [ \s{names} ]}
\dd
Displays all unproved conjectures [with the specified names].
\p
\dt \f{display facts [ \s{names} ]}
\dd
Displays all facts in the current proof context [with the specified names].
Indicates \dlink{../settings/immunity}{immune} facts by an \fq{(I)} following
their names, ancestor-immune statements by an \fq{(i)}, and
\dlink{../settings/activity}{passive} facts by a \fq{(P)}.
\p
\dt \f{display names}
\dd
Displays all \setlink{name-prefix}es introduced in the current proof context.
\p
\dt \f{display ordering-constraints [ \s{names} ]}
\dd
Displays the \dlink{register}{registry} for all operators [in the
named facts] in the current proof context, unless the value of the
\setlink{ordering-method} setting is \fq{polynomial}, in which case it displays
the \dlink{../ordering/polynomial}{polynomial interpretations} of these
operators. 
\p
\dt \f{display proof-status}
\dd
Displays the status of the current conjecture and all other conjectures for
which it is a subgoal (of a subgoal ...).
\p
\dt \f{display symbols [ \s{names} ]}
\dd
Displays all sorts, operators, and variables [in the named facts] in the
current proof context.  The command \fq{display symbols} displays all symbols
in the current proof context, whereas \fq{display symbols *} only displays
those that occur in some fact.
\end{description}
