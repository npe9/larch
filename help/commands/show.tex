\doc{The show command}
\ref{show-command}

The \def{show command} displays the results of certain operations without
affecting the state of LP's logical system.

\head{2}{\dlink{../symbols/syntax}{Syntax}}
\begin{verbatim}
\sd{show-command} ::= \f{show} \f{normal-form} \slink{../logic/term}{term}
                      | \f{show} \f{unifiers}  \slink{../logic/term}{term},  \slink{../logic/term}{term}
\end{verbatim}

Note: The first argument of the \fq{show} command can be
\dlink{../misc/abbreviation}{abbreviated}.

\head{2}{Examples}
\begin{verbatim}
show n-f e \in (s \U s)
show unifiers e*x, i(y)*y
\end{verbatim}

\head{2}{Usage} 
The \fq{show normal-form} command displays a \olink{normalization}{normal form}
of the term with respect to the currently \dlink{../settings/activity}{active}
rewrite rules.  If the \setlink{trace-level} is nonzero, LP also displays
successive reductions of the term leading to the normal form.
\p
The \fq{show unifiers} command displays a complete set of most general 
\olink{unify}{unifiers} of two terms.  It displays the unifying substitutions 
along with the unifications of the terms, and it uses unification algorithms
appropriate to the \llink{operator-theory}{theories} associated with the
operators in the terms.

