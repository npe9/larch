\doc{The unregister command}
\ref{unregister-command}
\ref{ordering-information}

The \def{unregister command} causes LP to discard information used to control
the registered and polynomial orderings.

\head{2}{\dlink{../symbols/syntax}{Syntax}}
\begin{verbatim}
\sd{unregister-command}   ::= \f{unregister} \s{ordering-information}
\sd{ordering-information} ::= \f{registry} | ( \f{bottom} | \f{top} ) \slink{../logic/operator}{operator}+[,]
\end{verbatim}

Note: The first word in the \s{ordering-information} can be
\dlink{../misc/abbreviation}{abbreviated}.


\head{2}{Examples}
\begin{verbatim}
unregister bottom succ
unregister registry
\end{verbatim}

\head{2}{Usage} 
The \fq{unregister registry} command deletes all information used by the
\dlink{../ordering/registered}{registered} and
\dlink{../ordering/polynomial}{polynomial} orderings to orient formulas into
rewrite rules.
\p
The \fq{unregister top} and \fq{unregister bottom} commands remove the listed
operators from the top and bottom of the registry used by the registered
orderings. 
