\doc{The lp-path setting}
\ref{lp-path-setting}
\ref{~lp}

The \def{lp-path setting} specifies a list of directories for LP to search when
looking for input from a file.

\head{2}{\dlink{../symbols/syntax}{Syntax}}
\begin{verbatim}
\sd{set-lp-path-command} ::= \f{set lp-path} \slink{../symbols/symbols}{string}
\end{verbatim}

\head{2}{Examples}
\begin{verbatim}
set lp-path . ~/myAxioms ~lp
\end{verbatim}

\head{2}{Usage}

The \fq{set lp-path} command specifies a search path for LP to use when looking
for help, \fq{.lp}, or \fq{.lpfrz} files.  Its default value is 
\qf{. ~lp/axioms}.  A period (\fq{.}) in the value of \fq{lp-path} refers to 
LP's current working \setlink{directory}.  A tilde (\fq{~}) in the value of 
\fq{lp-path} refers to the user's home directory.  An initial directory 
\fq{~lp} refers to the directory in which LP's runtime support was installed;
see the \dlink{../history/installation}{installation} instructions, the
\dlink{../misc/command-line}{command-line options}, and the \cflink{version} 
command for the location of this directory.


