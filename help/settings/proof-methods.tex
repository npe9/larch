\doc{The proof-methods setting}
\ref{proof-methods-setting}

The \def{proof-methods} setting provides a list of default proof methods for LP
to use when attempting to prove a formula.

\head{2}{\dlink{../symbols/syntax}{Syntax}}
\begin{verbatim}
\sd{set-proof-methods-command} ::= \f{set proof-methods} \slink{../proof/methods}{default-proof-method}+[,]
\end{verbatim}

Note: Each \s{default-proof-method} can be \dlink{../misc/abbreviation}{abbreviated}.

\head{2}{Examples}
\begin{verbatim}
set proof =>, normalization
\end{verbatim}

\head{2}{Usage}

The \fq{set proof-methods} command provides a list of default proof methods for
the current proof context.  LP uses the first method in the list that applies
to the conjecture.  The default list is \fq{normalization}.  Any method (other
than \fq{contradiction}) that does not mention a variable or constant can
appear on the list.  If the proof-method list is \fq{explicit-commands}, then
LP will await a \cflink{resume} command before beginning the proof.

\head{2}{See also}
\begin{itemize}
\item The \cflink{prove} command
\item The \dflink{../proof/of-conjunction}{/\-method}
\item The \dflink{../proof/of-implication}{=>-method}
\item The \dflink{../proof/of-biconditional}{<=>-method}
\item The \dflink{../proof/of-conditional}{if-method}
\item The \dflink{../proof/explicit}{explicit-commands} method
\item The \dflink{../proof/by-normalization}{normalization} method
\end{itemize}



