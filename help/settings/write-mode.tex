\doc{The write-mode setting}
\ref{write-mode-setting}
The \def{write-mode setting} controls the manner in which the \fq{write}
command outputs identifiers and terms.

\head{2}{\dlink{../symbols/syntax}{Syntax}}
\begin{verbatim}
\sd{set-write-mode-command} ::= \f{set write-mode} \slink{display-mode}{qualification-mode}
\end{verbatim}

\head{2}{Examples}
\begin{verbatim}
set write-mode qualified
\end{verbatim}

\head{2}{Usage}

The \fq{write-mode} setting controls the output of identifiers and terms by
the \cflink{write} command in the current proof context.
\begin{verbatim}
write-mode         effect
----------         ------
qualified          print qualifications for all subterms, identifiers
unqualified        print no qualifications
unambiguous        print enough qualifications to enable reparsing
\end{verbatim}
The default write-mode is \fq{qualified}, which guarantees that the output can
be reparsed even in the presence of additional \llink{overload}{overloadings}
for identifiers.  It is often desirable, however, to set the \fq{write-mode} to
\fq{unambiguous} to shorten and improve the readability of \fq{.lp} files.  If 
a problem arises in executing a \fq{.lp} file produced in this fashion (because
it is being executed in a context that overloads one of its operators), the
problem can be solved by starting a new copy of LP, executing the \fq{.lp}
file, and writing it out again in \fq{qualified} mode.

\head{2}{See also}
\begin{itemize}
\item \llink{overload}{Overloaded} identifiers
\item The \setlink{display-mode} setting
\end{itemize}
