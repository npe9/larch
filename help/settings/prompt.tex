\doc{The set prompt command}
\ref{prompt-setting}

The \def{prompt setting} defines a string that LP uses to prompt users to enter
a command.

\head{2}{\dlink{../symbols/syntax}{Syntax}}
\begin{verbatim}
\sd{set-prompt-command} ::= \f{set prompt} \s{prompt} 
\sd{prompt}             ::= \slink{../symbols/symbols}{blank-free-string} | \f{`} \slink{../symbols/symbols}{string} \f{'}
\end{verbatim}

\head{2}{Examples}
\begin{verbatim}
set prompt `[!] '
set prompt `>> '
\end{verbatim}

\head{2}{Usage}
By default, \s{prompt} is \f{`LP!: '}, which causes LP to issue prompts of the
form \qf{LP1: }, \qf{LP2: }, ...  To enter a prompt that begins or ends with a
space, enclose it within \fq{`'} marks.
\p
LP replaces the first exclamation mark (\fq{!}) in \s{prompt}, if any, by the
number of the next command.  LP numbers commands entered from the terminal by
consecutive integers.  It numbers commands obtained during execution of a
\setlink{script} (\f{.lp}) file by appending a period followed by consecutive 
integers to the number of the \cflink{execute} command; thus command 5.2.3 is
the third command in the script file executed in response to the second command
in the script file executed in response to the fifth command typed by the user.

