\doc{The hardwired-usage setting}
\ref{hardwired-usage-setting}

The \def{hardwired-usage setting} provides users with limited control over LP's
use of hardwired rewrite and deduction rules.

\head{2}{\dlink{../symbols/syntax}{Syntax}}
\begin{verbatim}
\sd{set-hardwired-usage-command} ::= \f{set hardwired-usage} \slink{../symbols/symbols}{number}
\end{verbatim}

\head{2}{Examples}
\begin{verbatim}
set hardwired-usage 8
\end{verbatim}

\head{2}{Usage} 
LP hardwires selected rewrite rules for the \llink{connective}{logical} and
\llink{conditional}{conditional} operators.  For debugging purposes, the
\fq{set hardwired-usage} command can be used to turn off some of these
hardwired rewrite rules in the current proof context.  The following powers of
2, if they occur in the binary representation of
\s{number}, turn off the indicated rule.
\begin{description}
\dt 2
\dd Turns off the rewrite rule \fq{x => false -> ~x}
\dt 4
\dd Turns off the rewrite rule \fq{x <=> false -> ~x}, but adds the rewrite
rule \fq{x <=> y <=> ~y -> ~x}.
\dt 8
\dd Turns off the rewrite rules with left side \fq{if x then y else z} when
\fq{y} or \fq{z} is \fq{true} or \fq{false}
\dt 16
\dd Turns off the if-simplification metarule
\end{description}
