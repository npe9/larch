\doc{The rewriting-limit setting}
\ref{rewriting-limit-setting}

The \def{rewriting-limit setting} sets an upper bound on the number of
reductions that LP will perform when normalizing a term with respect to a
rewriting system that is not guaranteed to terminate.

\head{2}{\dlink{../symbols/syntax}{Syntax}}
\begin{verbatim}
\sd{set-rewriting-limit-command} ::= \f{set rewriting-limit} \slink{../symbols/symbols}{number}
\end{verbatim}

\head{2}{Examples}
\begin{verbatim}
set rewriting-limit 50
\end{verbatim}

\head{2}{Usage}

The \fq{rewriting-limit} setting is local to the current proof context.  Its
default value is 1000.
\p
If LP exceeds the rewriting limit when \olink{normalization}{normalizing} a
formula, rewrite rule, or deduction rule, it
\dlink{../settings/immunity}{immunizes} that fact.  If it exceeds the rewriting
limit when attempting to prove a conjecture by normalization or rewriting, the
user can continue normalizing the conjecture by typing \cflink{resume} (after
raising the rewriting limit, if desired).

