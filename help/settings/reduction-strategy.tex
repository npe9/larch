\doc{The reduction-strategy setting}
\ref{reduction-strategy-setting}

The \def{reduction-strategy setting} controls the order in which LP applies
rewrite rules when reducing a term.

\head{2}{\dlink{../symbols/syntax}{Syntax}}
\begin{verbatim}
\sd{set-reduction-strategy-command} ::= \f{set reduction-strategy} 
                                        ( \f{inside-out} | \f{outside-in} )
\end{verbatim}

Note: The \fq{reduction-strategy} can be
\dlink{../misc/abbreviation}{abbreviated}.


\head{2}{Examples}
\begin{verbatim}
set reduction in
\end{verbatim}

\head{2}{Usage}
When the \fq{reduction-strategy} is \fq{outside-in} (the default), LP attempts
to \olink{reduce}{reduce} a term by attempting to rewrite the entire term
before attempting to reduce any of its subterms.  If it is \fq{inside-out}, LP
still applies the hardwired rewrite rules outside-in, but it attempts to apply
other rewrite rules to the subterms of a term before it applies them to the
entire term.
\p
The \fq{reduction-strategy} is local to the current proof context.
