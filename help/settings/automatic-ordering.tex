\doc{The automatic-ordering setting}
\ref{automatic-ordering-setting}

The \def{automatic-ordering} setting controls whether LP orients formulas into
rewrite rules with or without the user having to issue an explicit
\cflink{order} command.

\head{2}{\dlink{../symbols/syntax}{Syntax}}
\begin{verbatim}
\sd{set-automatic-ordering-command} ::= \f{set automatic-ordering} ( \f{on} | \f{off} )
\end{verbatim}

\head{2}{Examples}
\begin{verbatim}
set auto-ord off
\end{verbatim}

\head{2}{Usage}

If \fq{automatic-ordering} is \fq{on} (the default), then LP attempts to orient
all formulas (asserted axioms, proved conjectures, results of deductions,
instantiations, critical pair equations, and hypotheses for subgoals in proofs)
automatically into rewrite rules.  If it is \fq{off}, then the user must issue
an explicit \fq{order} command to orient formulas into rewrite rules.  The
value of this setting is local to the current proof context.
