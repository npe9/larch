\doc{The name-prefix setting}
\ref{name-prefix-setting}

The \def{name-prefix setting} specifies the identifier used to construct names
for newly asserted facts and conjectures.

\head{2}{\dlink{../symbols/syntax}{Syntax}}
\begin{verbatim}
\sd{set-name-prefix-command} ::= \f{set name-prefix} \slink{../symbols/symbols}{simpleId}
\end{verbatim}

\head{2}{Examples}
\begin{verbatim}
set name nat
\end{verbatim}

\head{2}{Usage}

The \fq{set name-prefix} command directs LP to use the \s{simpleId} as the
prefix for any new \dlink{../misc/name}{names} generated in the current proof
context.  After a command such \fq{set name nat}, LP assigns the names
\fq{nat.1}, \fq{nat.2}, ... in sequence to facts and conjectures.


