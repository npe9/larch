\doc{The log-file setting}
\ref{log-file-setting}
\ref{logging}

The \def{log-file setting} provides a means of recording the current LP session
in a file.

\head{2}{\dlink{../symbols/syntax}{Syntax}}
\begin{verbatim}
\sd{set-log-file-command} ::= \f{set log-file} \slink{../commands/execute}{file}
\end{verbatim}

\head{2}{Examples}
\begin{verbatim}
set log session
\end{verbatim}

\head{2}{Usage}

The \fq{set log} command causes LP to start recording the terminal session in a
file named \s{file}\f{.lplog} (unless \s{file} contains a period, in which case
LP does not supply the suffix \fq{.lplog}) in LP's current working
\setlink{directory}.  Any previous contents of this log file are lost.  If LP
was already logging to a file, that file is closed before opening the new log
file.  Logging is ended by the \cflink{quit} or
\dflink{../commands/unset}{unset log} commands.

