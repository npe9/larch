\doc{The page-mode setting}
\ref{page-mode-setting}

The \def{page-mode setting} controls whether or not LP pauses to enable users
to read its output before it is scrolled off the screen.


\head{2}{\dlink{../symbols/syntax}{Syntax}}
\begin{verbatim}
\sd{set-page-mode-command} ::= \f{set page-mode} ( \f{on} | \f{off} )
\end{verbatim}

\head{2}{Examples}
\begin{verbatim}
set page-mode on
\end{verbatim}

\head{2}{Usage}

When \fq{page-mode} is \fq{off} (the default), LP does not pause during output.
When it is \fq{on}, LP displays output a screen at a time.  After LP displays
each screenful of output, it prompts the user with \fq{--More--} to type a
character indicating what to do next.  The options are as follows:

\begin{verbatim}
Response  Action
--------  ------
\v{space}      display next screenful
\v{return}     display next line
\v{digit}      display next \v{digit} lines
d         display next half screenful
u         display continuously until next user interaction
q         display nothing until next user interaction
?         display this menu
\end{verbatim}

Most Unix systems also allow users to control output is by using the \fq{^S}
and \fq{^Q} keys; \fq{^S} stops output, and \fq{^Q} resumes printing.
