\doc{The completion-mode setting}
\ref{completion-mode-setting}

The \def{completion-mode setting} sets the operating mode for the 
\dlink{../commands/complete}{completion procedure}.

\head{2}{\dlink{../symbols/syntax}{Syntax}}
\begin{verbatim}
\sd{set-completion-mode-command} ::= \f{set completion-mode} \s{completion-mode}
\sd{completion-mode}             ::= \f{standard} | \f{expert} | \f{big}
\end{verbatim}

Note: The \s{completion-mode} can be \dlink{../misc/abbreviation}{abbreviated}.

\head{2}{Examples}
\begin{verbatim}
set completion-mode expert
\end{verbatim}

\head{2}{Usage}

The \fq{completion-mode} setting affects the order in which completion tasks
are executed in the current proof context.  It also affects the amount of user
interaction during the completion procedure.
\begin{itemize}
\item
The \def{standard} mode (the default) computes 
\dlink{../commands/critical-pairs}{critical pair equations} before extending 
the \dlink{../commands/register}{registry}.  When the
\setlink{automatic-registry} setting is \fq{off}, this reduces the amount of
interaction required from the user to extend the registry; however, it can be
inefficient.
\p
\item
The \def{expert} mode gives higher priority to orienting formulas into rewrite
rules than to computing critical pair equations.  When the
\setlink{automatic-registry} setting is \fq{off}, this gives the user more
explicit control over the completion process by prompting for guidance more
often.  In particular, user interaction to order unordered formulas is
requested before the system attempts to compute new critical pairs.
\p
\item
The \def{big} mode postpones the computation of critical pairs even farther,
so that big formulas are examined before critical pairs are computed.
\end{itemize}
