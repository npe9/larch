\doc{The display-mode setting}
\ref{display-mode-setting}
\ref{qualification-mode}

The \def{display-mode setting} controls the manner in which the \fq{display}
command outputs identifiers and terms.


\head{2}{\dlink{../symbols/syntax}{Syntax}}
\begin{verbatim}
\sd{set-display-command} ::= \f{set display-mode} \s{qualification-mode}
\sd{qualification-mode}  ::= \f{qualified} | \f{unambiguous} | \f{unqualified}
\end{verbatim}

Note: The \s{qualification-mode} can be \dlink{../misc/abbreviation}{abbreviated}.

\head{2}{Examples}
\begin{verbatim}
set display-mode qualified
\end{verbatim}

\head{2}{Usage}

The \fq{display-mode} setting controls the ouput of identifiers and terms by
the \cflink{display} command in the current proof context.
\begin{verbatim}
display-mode       effect
------------       ------
qualified          print qualifications for all subterms, identifiers
unqualified        print no qualifications
unambiguous        print enough qualifications to enable reparsing
\end{verbatim}
The default display-mode is \fq{unambiguous}.

\head{2}{See also}
\begin{itemize}
\item \llink{overload}{Overloaded} identifiers
\item The \setlink{write-mode} setting
\end{itemize}
