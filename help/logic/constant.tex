\doc{Constants}
\ref{constants}

A \def{constant} is a 0-ary \dlink{operator}{operator}.  An identifier for a
constant can be either a \slink{../symbols/symbols}{simpleId} (e.g., \f{0} and
\f{c}) or a \dlink{bracket}{bracketed} operator (e.g., \f#{}# and \f{[]}).  
\p
The same identifier can be \dlink{overload}{overloaded} to name two different
constants, that is, two constants with different sorts.  A \s{simpleId} can
also be used to name a \dlink{variable}{variable} and a constant, provided the
variable does not have the same sort as the constant.
\p
When LP formulates hypotheses for use in proving subgoals in a proof, it
generally replaces all free variables in the hypotheses by constants.  LP
creates identifiers for these constants by appending the letter \fq{c} and, if
necessary, further digits to obtain an identifier that is not already in use.
Thus LP may replace the variable \fq{x} by the constants \fq{xc}, \fq{xc1}, ...
