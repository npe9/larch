\doc{Equations}
\ref{equations}

An \def{equation} is a formula that consists of a pair of terms separated by an
\dlink{equality}{equality} operator.  

\head{2}{\dlink{../symbols/syntax}{Syntax}}
\begin{verbatim}
\sd{equation} ::= \slink{term}{term} (\f{=} | \f{<=>}) \slink{term}{term}
\end{verbatim}

Restriction: The two \s{term}s in an equation must have the same sort.  This
sort must be \f{Bool} if the equality operator is \f{<=>}.

\head{2}{Examples}
\begin{verbatim}
x + 0 = x
x <= y <=> x < y \/ x = y
x | y <=> \E z (y = x*z)
\end{verbatim}

\head{2}{Usage}
LP treats equations in the same manner as it treats formulas.  Indeed, any
formula is logically equivalent to an equation: \fq{F} is logically equivalent
to \fq{F <=> true}.
