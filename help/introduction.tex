\doc{Introduction}
\ref{LP}
\ref{introduction}

LP is an interactive theorem proving system for multisorted first-order logic.
It was developed by Stephen J. Garland and John V. Guttag at the MIT Laboratory
for Computer Science (a predecessor of CSAIL, the MIT Computer Science and
Artificial Intelligence Laboratory).
\p
Unlike many theorem provers, which attempt to find proofs automatically for
correctly stated conjectures, LP is intended to assist users in finding and
correcting flaws in conjectures---the predominant activity in the early stages
of the design process.  LP works efficiently on large problems, has many
important user amenities, and can be used by relatively naive users.
\p
LP was used at MIT and elsewhere primarily during the 1990s to reason about
designs for circuits, concurrent algorithms, hardware, and software.
\p
For a general introduction to LP, see the following topics.
\p
\begin{itemize}
\item \dlink{releases/installation}{Installing LP}
\item Using LP
\begin{itemize}
\item \dlink{misc/sample}{Some sample proofs}
\item \dlink{logic/logic}{Logical syntax and semantics}
\item \dlink{logic/system}{Operational syntax and semantics}
\item \dlink{proof/forward}{Forward inference mechanisms} for using axioms
\item \dlink{proof/backward}{Backward inference mechanisms} for proving conjectures
\item \dlink{commands/commands}{Commands} recognized by LP
\end{itemize}
\item \dlink{misc/philosophy}{Design philosophy}
\item \dlink{contents}{Table of contents for this documentation}
\end{itemize}
