\doc{Sample proofs: how to guide a proof}
Here are some things to try when LP and/or you get stuck trying to prove a
conjecture.  

\begin{itemize}
\item
Try a proof method based on the form of the conjecture.  For example, try
\fq{resume by <=>} if the conjecture is a logical equivalence.  Sometimes such
proof methods are useful ``no-brainers''.
\p
\item
Think about why you believe the conjecture.  
\p
\item
If you've forgotten where you are in a proof, type
\begin{verbatim}
\clink{display} proof    to see the current subgoal
display *Hyp     to see the current hypotheses
display          to see all facts
\clink{history}          to see how you got where you are
\end{verbatim}
\p
\item
Be skeptical: maybe the conjecture isn't a theorem.
\p
\item
Try a proof by cases, either to simplify the current subgoal or to make some
hypothesis more useful.
\p
\item
Look for a useful lemma to prove.
\p
\item
Try using the \clink{critical-pairs} command to derive consequences from the
hypotheses, for example, by typing \fq{critical-pairs *Hyp with *}.
\p
\item
Try alternative proof strategies.
\begin{itemize}
\item
The \clink{cancel} command lets you back up.
\item
The \clink{freeze} and thaw commands let you save LP's state in a file and get
it back again.
\end{itemize}
\p
\item
Think again about why you still believe the conjecture.
\p
\item
Try explaining why LP must be broken to a colleague.  (The colleague does not
need to understand anything about LP.  He or she simply needs to appear to be
listening attentively.)
\p
\item
If you still believe that LP is broken, send e-mail to \fq{larch@lcs.mit.edu}.
\end{itemize}
