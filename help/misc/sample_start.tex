\doc{Sample proofs: getting started}

To invoke LP, type \fq{lp} at the Unix command prompt.  LP will respond with
the following message:
\begin{verbatim}
Welcome to LP (the Larch Prover), Release 3.1 (94/12/30).
Copyright (C) 1994, S. J. Garland and J. V. Guttag
Type `help lp' (followed by a carriage return) for help.

LP1:
\end{verbatim}
LP is a command-driven system.  The last line of LP's welcoming message,
\fq{LP1:}, is a \setlink{prompt} for the first command.  LP prompts users to
enter subsequent commands with \fq{LP2:}, \fq{LP3:}, ...
\p
You can type input directly in response to LP's prompts, or you can create a
file of LP commands (e.g., \dflink{set1.lp}{set1.lp}) and then use LP's
\clink{execute} command to cause LP to execute the commands in that file as if 
you had typed them directly.  Here's the start of a session in which commands
are executed from the file \fq{set1.lp}:
\begin{verbatim}
LP1: execute set1

LP1.1: % Some simple theorems about finite sets

LP1.2:
\end{verbatim}
The first command, \fq{execute set1}, was typed by the user.  The second was
obtained from the file \fq{set1.lp}.  This command is a \clink{comment}: LP
ignores all input following a percent sign.  The prompts \fq{LP1.1:} and
\fq{LP1.2:} indicate that LP is taking input from the file being executed as a
result of the command entered in response to the prompt \fq{LP1:}.
