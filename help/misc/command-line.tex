\doc{Unix command-line options}
\ref{command-line}

The following options can be specified on the Unix command line when invoking LP:
\begin{description}
\dt \f{-c}
\dd
Enables experimental features for conditional rewriting.
\p
\dt \f{-d} \v{fileName}
\dd
Specifies the location of LP's run-time library, to which \fq{~lp} in the
\setlink{lp-path} setting refers.  The default is \fq{/usr/local/lib/LP}.
\p
\dt \fq{-e}
\dd
Enables undocumented experimental features in LP.
\p
\dt \fq{-max_heap} \s{number}
\dd
Sets an upper bound, in megabytes, on the size of LP's heap.  This bound should
be large enough for LP top handle a proof without running its collector too
often (for example, 10 meg on a 32-bit machine and 20-meg on a 64-bit machine).
It should be small enough for LP to run without paging (for example, half of
the total amount of memory on a single-user machine).

Improvements in garbage collection, as well as increases in memory sizes, that
have occurred since LP's initial development make this and the \fq{-min_gc}
command-line option less useful than before.  Hence they are being deprecated
and may not be available in recent distributions of LP.  
\p
\dt \fq{-min_gc} \s{number}
\dd
Sets the minimum threshhold, in megabytes, beneath which LP's collector will
not run automatically.
\p
\dt \fq{-t}
\dd
Prevents LP from aborting execution of \fq{.lp} files when an error occurs;
useful for testing LP.
\end{description}
