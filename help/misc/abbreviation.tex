\doc{Abbreviations for commands}
\ref{abbreviations}

LP allows users to abbreviate a command with an unambiguous prefix of that
command.  For example, the \fq{declare} command can be abbreviated to \fq{dec},
but not to \fq{de} because \fq{de} is also a prefix of the \fq{delete} command.
\p
LP also allows users to abbreviate certain command arguments with unambiguous
prefixes.  When the argument can be a hyphenated word, it is enough to supply
unambiguous prefixes of the hyphenated components.  Examples:
\begin{verbatim}
Command                              Abbreviation
-------                              ------------
declare operators 0, 1: -> Nat       dec op 0, 1: -> Nat
display rewrite-rules                dis r-r
\end{verbatim}
When an argument to a command can contain reserved words (such as
\fq{formulas}) and \dlink{name}{names} of facts, abbreviations used for 
reserved words must not conflict with any \setlink{name-prefix} established by
the \cflink{set} command.  For example, \fq{disp form} is a legal abbreviation
for the command \fq{display formulas} unless the user has previously entered
the command \fq{set name form}, in which case LP interprets \fq{disp form} as
a request to display all facts whose name begins with the prefix \fq{form}
rather than to display all facts that are formulas.

