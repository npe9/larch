\doc{Registered orderings}
\ref{registered-ordering}
\ref{suggestions}

LP provides two \def{registered orderings}, the 
\dlink{dsmpos}{dsmpos and noeq-dsmpos orderings}, which are based on 
LP-suggested partial orderings of operators and which guarantee termination of
the set of rewrite rules when no
\llink{operator-theory}{associative-commutative} operators are present.  Most
users rely on these orderings to orient all formulas.  In striking contrast to
the \dlink{brute-force}{brute-force} orderings, they hardly ever cause
difficulties by producing a nonterminating set of rewrite rules.
\p
Registered orderings use information in a \def{registry} to orient formulas
into rewrite rules.  There are two kinds of information in a registry:
\dlink{height}{height} information and \dlink{status}{status} information.
\p
When the current registry does not contain enough information to orient a
formula, LP will generate minimal sets of extensions to the registry, called
\def{suggestions}, that permit the formula to be oriented.  If the
\setlink{automatic-registry} setting is \fq{on}, LP picks one of these 
extensions to orient a formula without user interaction; it does not try all
extensions.  If the setting is \fq{off}, LP will \dlink{interactive}{interact}
with the user, who must pick the desired extension.

