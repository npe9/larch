\doc{Release 3.1b}
\ref{release3_1b}

The following bugs in Version 3.1a have been fixed in Version 3.1b.
\begin{description}
\dt 95-014
\dd
The \fq{normalize} command now properly normalizes a rewrite rule with a
reducible left side.  Formerly, it failed to normalize a rewrite rule when 
its left side could be reduced, but no further reductions were possible.
\dt 95-015
\dd
The operation of boolean equality is now recognized as an ac operator and not
just as a commutative operator.  Formerly, LP incorrectly (and unsoundly)
eliminated some subterms when reducing formulas such as
\fq{a = (r = (b = b'))} that contain multiple occurrences of equality
between boolean subformulas.
\dt 95-016
\dd
The experimental version of LP (invoked by the command \fq{lp -e}) no longer
crashes when a \fq{register} command is issued during  a proof.
\dt 98-001
\dd
The bound variable \fq{i1} in
\begin{quote}
\fq{f(i, j) <=> \A i1 \A j1 (i ~= i1 \/ j ~= j1 => P(i1, j1))}
\end{quote}
is no longer changed incorrectly to \fq{i2} when executing
\fq{show normal-form f(i2, i1)}.
\end{description}

In addition, the following improvements have been made in Version 3.1b.

\begin{itemize}
\item
LP now contains an experimental implementation of an enhanced dsmpos ordering,
which can be activated by the \fq{lp -debug} command and used to prove the
termination of rewriting systems in the presence of ac operators.  The enhanced
ordering uses a variant, developed in collaboration with Deepak Kapur, of the
multiset comparison.
\item
The command \fq{display conjectures \s{names}} now shows only the current
conjecture when no names are specified.  To show all named conjectures, the
user must now type \fq{display conjectures (*)}.  The parentheses are required,
lest the command be parsed as \fq{display conjectures*}.
\item
The class expressions \fq{contains-operator(\s{operator>})} and
\fq{contains-variable(\s{variable})} now allow an unqualified operator
or variable to be overloaded, in which case the class expression names all
facts containing at least one of the overloading.
\item
The routines that display the precedence relation (for the \fq{write} and
\fq{display} commands) now display a generating subset of the relation,
not the entire relation.
\item
There is now a class expression \fq{contains-match(\s{term})}, which
matches all facts that contain a subterm matching the given term.
\end{itemize}

