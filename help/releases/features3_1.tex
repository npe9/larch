\doc{New features in Release 3.1}
\ref{features-new}

The following features have been added to LP since Release 2.4.
\begin{itemize}
\item
Support for full first-order logic, not just the universal-existential subset
supported by Release 2.4.  See \llink{quantifier}{quantifiers}.
\p
\item
A simple \llink{sort}{sort} system for describing polymorphic abstractions.
\p
\item
New inference mechanisms
\begin{itemize}
\item
Proofs by well founded \plink{by-induction}{induction}.
\item
Proofs of conjectures of the form \fq{t1 <=> t2}, with \fq{t1 => t2}
and \fq{t2 => t1} as subgoals.
\item
Proofs that use deduction rules for backward inference.  See \clink{apply}.
\end{itemize}
\p
\item
Richer syntactic conventions, such as \fq{x[n]} and \fq{n!}, for 
\llink{operator}{operators} and terms.
\p
\item
Additional user amenities, for example, enhanced facilities for
\dlink{../misc/names}{naming} sets of statements.
\p
\item
New rewrite rule for the boolean operators, namely, \fq{~p <=> ~q -> p <=> q}.
\end{itemize}

The following features behave differently in Release 3.1 than they did in
Release 2.4.

\begin{itemize}
\item
Some \dlink{../commands/set}{settings} (e.g., the default name prefix) are
now local to proof contexts.
\p
\item
Names such as \f{thmCaseHyp1.1} are reused within non-overlapping proof
contexts.
\p
\item
LP now attempts to preserve the order of operands in formulas such as 
\fq{init => c = 1}, so that preference is given to ordering equalities in the 
direction the user may have intended when these formulas normalize to
equations.
\end{itemize}

