\doc{Development History}
\ref{development}

LP was developed by Stephen J. Garland and John V. Guttag at the MIT Laboratory
for Computer Science, a predecessor of CSAIL, the MIT Computer Science and
Artificial Intelligence Laboratory.  The initial implementation of LP was based
on Reve, a rewrite rule laboratory developed by Pierre Lescanne with assistance
from Randy Forgaard, David Detlefs, and Katherine Yelick.

\head{2}{Releases}
Several versions, distinguished by their release dates, exist for the early
releases of LP.

\head{4}{Release 1.0, 1988-1990}
The first release of LP extended Reve's term-rewriting system for equational
logic to a proof system for quantifier-free first-order logic.  It provided
\begin{itemize}
\item
mechanisms for proofs by induction, cases, and contradiction,
\item
user-defined deduction rules, and
\item
unification and matching for commutative operators in addition to Reve's
facilities for associative-commutative operators.
\end{itemize}

\head{4}{Release 2.0, December 27, 1990}
This release contained improvements to the code in Version 1.0.  It also provided
\begin{itemize}
\item
a simpler and more uniform command syntax,
\item
uniform mechanisms for naming, asserting, and proving all components of its
logical system,
\item
hardwired axioms for logical operators, and
\item
improved performance.
\end{itemize}

\head{4}{Release 2.1, August 21, 1991}
This release cleaned up many of the features introduced in Release 2.0 and made
them more uniform.

\head{4}{Release 2.2, November 17, 1991}
This release provided enhanced deductive mechanisms and better performance.  It
is the one documented in \def{A Guide to LP, the Larch Prover}, by Stephen
J. Garland and John V. Guttag, published on December 31, 1991, as \def{Report 82}
by the System Research Center of the Digital Equipment Corporation.

\head{4}{Release 2.2a, June 4, 1992}
This release provided an experimental implementation of conditional rewriting, 

\head{4}{Release 3.1, December 30, 1994}
Release 3.1 was a major new release that  extended LP's proof system to one for
full mulstisorted first-order logic.  For more details, see the lists of
\begin{itemize}
\item 
\dlink{features3_1}{features} added in Release 3.1, ad
\item 
\dlink{changes3_1}{changes} required in old proof scripts for use with
Release 3.1.
\end{itemize}

\head{4}{Release 3.1a, April 27, 1995}
This release fixed some \dlink{release3_1a}{bugs in Release 3.1}.

\head{4}{Release 3.1b, September 5, 1997, through January 28, 1999}
Release 3.1b was the final official release of LP.  It fixed
some \dlink{release3_1b}{bugs in Release 3.1a} and added a few new features. 

\head{2}{Support}

LP's development was supported by the following grants from the National Science
Foundation and DARPA, the Defense Advanced Research Projects Agency.  Additional
support was provided by the Systems Research Center of the Digital Equipment
Corporation.
\begin{itemize}
\item NSF CCR-8706652 (July 1, 1987 to December 31, 1989), ``Automated semantic
analysis of formal specifications.''
\item NSF CCR-8910848 (July 15, 1989, to December 31, 1991), ``Formal
specification of program module interfaces.''
\item NSF INT-9016780, US-France Cooperative Science Program (April 1, 1991, to
September 30, 1994), ''Integrating a theorem prover and a specification
environment.'' 
\item DARPA N00014-92-J-1795 (January 1, 1992, to March 31, 1995), ``Practical
applications of formal specifications.''
\item NSF CCR-9115797 (March 1, 1992, to August 31, 1995), ``Automated reasoning
in software engineering.''
\item NSF CCR-9504248 (August 1, 1995, to July 31, 1998), ``Automated reasoning
in software engineering.''
\end{itemize}
