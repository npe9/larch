\doc{Normalization}
\ref{normalize}
\ref{normalization}
\ref{normal-form}

A term is said to be \def{normalized}, or in \def{normal form}, with respect to
a set of \llink{rewrite-rule}{rewrite rules} if none of those rules can be
applied to the term.  The command \dflink{../commands/show}{show normal-form}
displays the normal form of a term with respect to the active rewrite
rules in LP's logical system.  See also the \cflink{normalize} command.
\p
If the rewriting system is not guaranteed to terminate (e.g., if the user
oriented an equation into a rewrite rule manually), the normal form computation
may not terminate.  When the rewriting system is not known to terminate, LP
stops the rewriting process and issues a warning after a very large number of
rewrites during a normal form computation.  See the \setlink{rewriting-limit}
setting.
\p
If the rewriting system is not \glink{confluent}{confluent}, a term
may have more than one normal form.
\p
LP uses normalization as a method of \plink{forward}{forward} and
\plink{backward}{backward} inference.
