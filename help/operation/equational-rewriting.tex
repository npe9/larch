\doc{Equational term rewriting}
\ref{equational-term-rewriting}
\ref{equational-rewriting}
\ref{equational-theory}

\def{Equational term-rewriting} differs from conventional term-rewriting in 
that \dlink{match}{matching}, \dlink{unify}{unification}, and 
\dlink{../commands/complete}{completion} are performed module a set \f{E} of
equations.
\p
A set \f{E} of equations defines an \def{equational theory}, which is the set
of equations that can be derived from \f{E} by substituting equals for equals.
More generally, a set of \llink{equation}{equations} and 
\llink{rewrite-rule}{rewrite rules} defines an equational
theory, which is that obtained by considering the rewrite rules as equations.
A major purpose of the Knuth-Bendix completion procedure is to provide a
decision procedure (reduction to normal form) for the question of whether
equations are in the equational theory of a given system of equations and
rewrite rules.
